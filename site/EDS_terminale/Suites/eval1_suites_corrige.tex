% filepath: /home/alaeddine/Documents/Interface__Maths_2025_2026/Interface_Maths_2025_2026/site/EDS_terminale/Suites/eval1_suites_corrige.tex
\documentclass[14pt, a4paper]{extarticle}
\usepackage[utf8]{inputenc}
\usepackage[T1]{fontenc}
\usepackage[french]{babel}
\usepackage{lmodern}
\usepackage{geometry}
\usepackage{amsmath, amssymb}
\usepackage{enumitem}
\usepackage{xcolor}
\usepackage[skins]{tcolorbox}

\geometry{margin=2cm}
\setlist[enumerate,1]{label=\textbf{\arabic*)}, itemsep=5pt}
\setlist[itemize,1]{label=\textbullet, itemsep=2pt, leftmargin=*}

\tcbset{
    colback=gray!5,
    colframe=black!60,
    boxrule=0.5pt,
    arc=2mm,
    title style={colback=black!75, coltext=white}
}

\newcommand{\R}{\mathbb{R}}
\newcommand{\N}{\mathbb{N}}

\begin{document}

\begin{center}
    \Large\textbf{Corrigé détaillé et barème}\\[0.2cm]
    \large Terminale EDS Mathématiques — Suites et récurrence \\
    Mardi 23 septembre 2025
\end{center}

\vspace{0.5cm}

% =============================
% CORRECTIONS DETAILLEES
% =============================

\section*{Exercice 1 (5 points) -- Monotonie par récurrence}
On considère la suite $(u_n)$ définie par $u_0 = 1$ et pour tout $n \in \N$, $u_{n+1} = \dfrac{u_n}{2} + 3$.

\begin{enumerate}
    \item \textbf{Calcul de $u_1$ et $u_2$} : (1 pt)
    \[ u_1 = \frac{u_0}{2} + 3 = \frac{1}{2} + 3 = \frac{7}{2} \qquad u_2 = \frac{u_1}{2} + 3 = \frac{7/2}{2} + 3 = \frac{19}{4} \]

    \item \textbf{Démontrer par récurrence que la suite $(u_n)$ est croissante.}
    Soit $P(n)$ la propriété : $u_{n+1} \ge u_n$.
    \begin{itemize}
        \item \textbf{Initialisation} (pour $n=0$) : (0.5 pt)
        On a $u_1 = \frac{7}{2}$ et $u_0 = 1$. Comme $u_1 > u_0$, la propriété $P(0)$ est vraie.

        \item \textbf{Hérédité} : (2.5 pts)
        Supposons que $P(k)$ est vraie pour un certain entier $k \ge 0$, c'est-à-dire $u_{k+1} \ge u_k$.
        On veut montrer que $u_{k+2} \ge u_{k+1}$.
        \[ u_{k+2} - u_{k+1} = \left(\frac{u_{k+1}}{2} + 3\right) - \left(\frac{u_k}{2} + 3\right) = \frac{1}{2}(u_{k+1} - u_k) \]
        Par hypothèse de récurrence, $u_{k+1} - u_k \ge 0$, donc $\frac{1}{2}(u_{k+1} - u_k) \ge 0$.
        Ainsi, $u_{k+2} - u_{k+1} \ge 0$, ce qui signifie que $P(k+1)$ est vraie.

        \item \textbf{Conclusion} : (1 pt)
        La propriété est vraie pour $n=0$ et est héréditaire, donc par le principe de récurrence, la suite $(u_n)$ est croissante.
    \end{itemize}
\end{enumerate}

\section*{Exercice 2 (5 points) -- Conjecture et démonstration}
Soit la suite $(v_n)$ définie par $v_0 = 0$ et pour tout $n \in \N$, $v_{n+1} = v_n + \dfrac{1}{2^n}$.

\begin{enumerate}
    \item \textbf{Calcul des premiers termes} : (1.5 pt)
    $v_1 = 1$, $v_2 = \frac{3}{2}$, $v_3 = \frac{7}{4}$.

    \item \textbf{Démontrer par récurrence que $v_n = 2 - \frac{2}{2^n}$.}
    \begin{itemize}
        \item \textbf{Initialisation} (pour $n=0$) : (0.5 pt)
        $v_0 = 0$. La formule donne $2 - \frac{2}{2^0} = 2 - 2 = 0$. Vrai.
        \item \textbf{Hérédité} : (2.5 pts)
        Supposons $v_k = 2 - \frac{2}{2^k}$ pour un $k \ge 0$.
        \[ v_{k+1} = v_k + \frac{1}{2^k} = \left(2 - \frac{2}{2^k}\right) + \frac{1}{2^k} = 2 - \frac{1}{2^k} = 2 - \frac{2}{2^{k+1}} \]
        La propriété est donc vraie au rang $k+1$.
        \item \textbf{Conclusion} : (0.5 pt)
        La formule est vraie pour tout $n \in \N$.
    \end{itemize}
\end{enumerate}

\section*{Exercice 3 (5 points) -- Suite arithmético-géométrique}
Soit la suite $(w_n)$ définie par $w_0 = 2$ et pour tout $n \in \N$, $w_{n+1} = 3w_n - 2$.

\begin{enumerate}
    \item \textbf{Calcul des premiers termes} : (1 pt)
    $w_1 = 4$, $w_2 = 10$.

    \item \textbf{Démontrer par récurrence que $w_n = 3^n + 1$.}
    \begin{itemize}
        \item \textbf{Initialisation} (pour $n=0$) : (0.5 pt)
        $w_0 = 2$. La formule donne $3^0 + 1 = 2$. Vrai.
        \item \textbf{Hérédité} : (3 pts)
        Supposons $w_k = 3^k + 1$ pour un $k \ge 0$.
        \[ w_{k+1} = 3w_k - 2 = 3(3^k + 1) - 2 = 3^{k+1} + 3 - 2 = 3^{k+1} + 1 \]
        La propriété est donc vraie au rang $k+1$.
        \item \textbf{Conclusion} : (0.5 pt)
        La formule est vraie pour tout $n \in \N$.
    \end{itemize}
\end{enumerate}

\section*{Exercice 4 (5 points) -- Suite et carrés}
Soit la suite $(p_n)$ définie par $p_0 = 1$ et pour tout $n \in \N$, $p_{n+1} = p_n + 2n + 1$.

\begin{enumerate}
    \item \textbf{Calcul des premiers termes} : (1.5 pt)
    $p_1 = 2$, $p_2 = 5$, $p_3 = 10$.

    \item \textbf{Démontrer par récurrence que $p_n = n^2 + 1$.}
    \begin{itemize}
        \item \textbf{Initialisation} (pour $n=0$) : (0.5 pt)
        $p_0 = 1$. La formule donne $0^2 + 1 = 1$. Vrai.
        \item \textbf{Hérédité} : (2.5 pts)
        Supposons $p_k = k^2 + 1$ pour un $k \ge 0$.
        \[ p_{k+1} = p_k + 2k + 1 = (k^2 + 1) + (2k + 1) = k^2 + 2k + 2 \]
        Or, la formule au rang $k+1$ donne $(k+1)^2 + 1 = (k^2 + 2k + 1) + 1 = k^2 + 2k + 2$.
        L'égalité est vérifiée, donc la propriété est vraie au rang $k+1$.
        \item \textbf{Conclusion} : (0.5 pt)
        La formule est vraie pour tout $n \in \N$.
    \end{itemize}
\end{enumerate}

\newpage

% =============================
% FICHE RAPIDE POUR LES CORRECTEURS
% =============================

%\section*{Fiche rapide pour les correcteurs}
%Chaque exercice vaut 5 points. Total : 20 points.

\begin{tcolorbox}[title=Exercice 1 (5 points)]
\begin{itemize}
    \item $u_1, u_2$ corrects : 1 pt.
    \item Initialisation : 0.5 pt.
    \item Hérédité : 2.5 pts.
    \item Conclusion : 1 pt.
    \item \textbf{Pénalités} : absence de conclusion ($-0.5$), absence d’hérédité ($-2.5$), erreurs mineures ($-0.25$).
\end{itemize}
\end{tcolorbox}

\begin{tcolorbox}[title=Exercice 2 (5 points)]
\begin{itemize}
    \item $v_1, v_2, v_3$ corrects : 1.5 pt.
    \item Initialisation : 0.5 pt.
    \item Hérédité : 2.5 pts.
    \item Conclusion : 0.5 pt.
    \item \textbf{Pénalités} : un terme faux ($-0.5$), conclusion manquante ($-0.5$), erreur algébrique ($-0.5$).
\end{itemize}
\end{tcolorbox}

\begin{tcolorbox}[title=Exercice 3 (5 points)]
\begin{itemize}
    \item $w_1, w_2$ corrects : 1 pt.
    \item Initialisation : 0.5 pt.
    \item Hérédité : 3 pts.
    \item Conclusion : 0.5 pt.
    \item \textbf{Pénalités} : absence d’initialisation ($-0.5$), erreur de calcul ($-0.25$), conclusion manquante ($-0.5$).
\end{itemize}
\end{tcolorbox}

\begin{tcolorbox}[title=Exercice 4 (5 points)]
\begin{itemize}
    \item $p_1, p_2, p_3$ corrects : 1.5 pt.
    \item Initialisation : 0.5 pt.
    \item Hérédité : 2.5 pts.
    \item Conclusion : 0.5 pt.
    \item \textbf{Pénalités} : un terme faux ($-0.5$), absence d’initialisation ($-0.5$), pas de conclusion ($-0.5$).
\end{itemize}
\end{tcolorbox}



\end{document}
