\documentclass[a4paper, 12pt]{article}
% \usepackage[utf8]{inputenc} % Plus nécessaire avec XeLaTeX/LuaLaTeX
% \usepackage[T1]{fontenc}    % Plus nécessaire avec XeLaTeX/LuaLaTeX
\usepackage{fontspec}         % Pour gérer les polices et les caractères Unicode
\usepackage[french]{babel}
% \usepackage{lmodern}        % Remplacé par la gestion de fontspec
\usepackage{geometry}
\usepackage{xcolor}
\usepackage[most]{tcolorbox}
\usepackage{enumitem}

\geometry{margin=2cm}
\pagestyle{empty} % Pour ne pas avoir de numéro de page


\begin{document}

\begin{center}
    {\Huge\bfseries 🌟 Autocorrection entre pairs : pourquoi ? 🌟}\\[1cm]
    {\Large Terminale EDS Mathématiques – Première évaluation}
\end{center}

\vspace{0.8cm}

\begin{tcolorbox}[colback=blue!5, colframe=blue!70!black, title=\textbf{1. Comprendre ce qui est attendu}]
Corriger la copie d’un camarade, c’est voir très concrètement :
\begin{itemize}[leftmargin=*]
    \item ce qu’est une rédaction claire et complète ;
    \item les erreurs fréquentes à éviter ;
    \item pourquoi une étape vaut des points, même sans avoir la réponse finale.
\end{itemize}
\end{tcolorbox}

\begin{tcolorbox}[colback=green!5, colframe=green!60!black, title=\textbf{2. Apprendre à s’auto-évaluer}]
En corrigeant, tu développes ton regard critique : tu vois ce qui est juste, tu repères les erreurs… et tu reconnais mieux les tiennes. Cela te rend plus autonome et plus fort pour progresser.
\end{tcolorbox}

\begin{tcolorbox}[colback=orange!5, colframe=orange!80!black, title=\textbf{3. Coopérer plutôt que rivaliser}]
Pas de compétition ici : tu aides ton camarade à comprendre ses erreurs, et il t’aide à comprendre les tiennes. Ensemble, vous progressez et vous avancez.
\end{tcolorbox}

\begin{tcolorbox}[colback=purple!5, colframe=purple!70!black, title=\textbf{4. Gagner en confiance}]
Cette première évaluation n’est pas là pour mettre la pression, mais pour apprendre. Corriger toi-même te montre que tu es capable d’expliquer et de comprendre, même si tu as fait des erreurs.
\end{tcolorbox}

\begin{tcolorbox}[colback=red!5, colframe=red!70!black, title=\textbf{5. Construire de bonnes habitudes}]
Grâce à cette autocorrection, tu prends de bonnes habitudes pour toute l’année (et pour le bac !) :
\begin{itemize}[leftmargin=*]
    \item toujours rédiger clairement ;
    \item toujours justifier tes réponses ;
    \item toujours suivre une démarche complète et rigoureuse.
\end{itemize}
\end{tcolorbox}

\vspace{0.8cm}

\begin{center}
    {\LARGE\bfseries 👉 Tu n’es pas seulement corrigé… tu deviens acteur de ta progression !}
\end{center}

\end{document}
