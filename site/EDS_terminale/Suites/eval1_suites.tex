\documentclass[12pt,a4paper]{article}
\usepackage[utf8]{inputenc}
\usepackage[T1]{fontenc}
\usepackage{lmodern}
\usepackage{geometry}
\usepackage{amsmath, amssymb}
\usepackage{multicol}

\geometry{margin=2.5cm}

\begin{document}

\begin{center}
\Large\textbf{Lycée Pierre Mendès France -- Tunis}\\[0.2cm]
\large Terminale EDS Mathématiques -- Groupe 3 \\[0.2cm]
\large \textbf{Évaluation : Suites et récurrence}\\[0.2cm]
Mardi 23 septembre 2025 -- Durée : 1 heure
\end{center}

\vspace{0.5cm}

\noindent\textbf{Consignes :}
\begin{itemize}
    \item La rédaction doit être \textbf{claire, complète et rigoureuse}. Chaque résultat doit être justifié.
    \item Écrire de manière lisible et aérée.
    \item Utiliser de préférence un \textbf{stylo noir}, ou à défaut un \textbf{stylo bleu}.
    \item Toute tentative de brouillon illisible ou de rédaction bâclée sera pénalisée.
\end{itemize}

\vspace{0.5cm}

\hrule
\vspace{0.5cm}

\section*{Exercice 1 (5 points) -- Monotonie par récurrence}
On considère la suite $(u_n)$ définie par :
\[
u_0 = 1, \quad u_{n+1} = \tfrac{u_n}{2} + 3 \quad \text{pour tout } n \in \mathbb{N}.
\]

\begin{enumerate}
    \item Calculer $u_1$ et $u_2$.
    \item Démontrer par récurrence que $u_{n+1} \geq u_n$ pour tout $n \in \mathbb{N}$.
\end{enumerate}

\vspace{0.5cm}

\section*{Exercice 2 (5 points)}
On considère la suite $(v_n)$ définie par :
\[
v_0 = 0, \quad v_{n+1} = v_n + \tfrac{1}{2^n} \quad \text{pour tout } n \in \mathbb{N}.
\]

\begin{enumerate}
    \item Calculer $v_1, v_2, v_3$.
    \item Démontrer par récurrence que, pour tout $n \in \mathbb{N}$, on a $v_n =2-\dfrac{1}{2^{n-1}}$.
\end{enumerate}

\vspace{0.5cm}

\section*{Exercice 3 (5 points) -- Expression explicite par récurrence}
On considère la suite $(w_n)$ définie par :
\[
w_0 = 2, \quad w_{n+1} = 3w_n - 2 \quad \text{pour tout } n \in \mathbb{N}.
\]

\begin{enumerate}
    \item Calculer $w_1$ et $w_2$.
    \item Démontrer par récurrence que, pour tout $n$, $w_n = 1+ 3^n$.
\end{enumerate}

\vspace{0.5cm}

\section*{Exercice 4 (5 points) -- Propriété numérique}
On définit la suite $(p_n)$ par :
\[
p_0 = 1, \quad p_{n+1} = p_n + 2n + 1.
\]

\begin{enumerate}
    \item Calculer $p_1, p_2, p_3$.
    \item Démontrer par récurrence que, pour tout $n \in \mathbb{N}$, $p_n = n^2+1$.
\end{enumerate}

\vspace{0.8cm}
\hrule
\vspace{0.3cm}

\noindent\textbf{Barème indicatif :}\\
Ex.1 : 5 pts \quad Ex.2 : 5 pts \quad Ex.3 : 5 pts \quad Ex.4 : 5 pts \\
\textbf{Total : 20 points}

\end{document}
q
