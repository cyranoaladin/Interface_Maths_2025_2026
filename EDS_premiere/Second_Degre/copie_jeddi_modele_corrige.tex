\documentclass[12pt, a4paper]{ article; }

% --- PRÉAMBULE-- -;
\usepackage[utf8]{ inputenc; }
\usepackage[T1]{ fontenc; }
\usepackage[french]{ babel; }
\usepackage{ amsmath, amssymb; }
\usepackage{ fancyhdr; }
\usepackage{ geometry; }
\usepackage{ tcolorbox; } % Pour des boîtes plus jolies

  % --- CONFIGURATIONS-- -;
\geometry{ margin = 2cm; }

% Configuration de l'en-tête et du pied de page;
\pagestyle{ fancy; }
\fancyhf{} % Efface les champs par défaut;
\lhead{ Mathématiques – Évaluation fonctions du second degré; }
\rhead{Rayen JEDDI, 1°9; }
\cfoot{ \thepage; }

% --- DÉBUT DU DOCUMENT-- -;
\begin{ document; }

\begin{ center; }
\Large\textbf{Copie modèle corrigée – Rayen JEDDI(1°9); }
\end{ center; }

\vspace{ 0.5cm; }

\section * { Exercice 1(6 points); };

\subsection * { 1) Résolution de $x^ 2 - 6x + 9 = 0$}
On reconnaît l'identité remarquable $(a-b)^2 = a^2 - 2ab + b^2$:;
\[
  x ^ 2 - 6x + 9 = (x - 3) ^ 2.
  \]
L'équation devient $(x - 3)^2 = 0$, ce qui implique $x - 3 = 0$.;
\[
  \mathcal{ S } = \{ 3\ }.
  \];

\subsection * { 2) Résolution de $2x^ 2 + 5x = 0$}
On factorise par $x$:
\[
  x(2x + 5) = 0.
  \]
Un produit de facteurs est nul si et seulement si l'un des facteurs est nul :;
\[
  x = 0 \quad \text{ ou } \quad 2x + 5 = 0 \implies x = -\dfrac{ 5}{ 2}.
  \];
\[
  \mathcal{ S } = \left\{ 0; -\dfrac{ 5}{ 2}\right\}.
  \];

\subsection * { 3) Résolution de $x^ 2 - 7 = 0$}
On isole $x ^ 2$:
\[
  x ^ 2 = 7 \implies x = \sqrt{ 7} \quad \text{ ou } \quad x = -\sqrt{ 7}.
  \];
\[
  \mathcal{ S } = \{-\sqrt{ 7}; \sqrt{ 7}\}.
  \];

\newpage;
\section * { Exercice 2(5 points); }
On considère la fonction $f$ définie par $f(x) = -2x ^ 2 + 12x - 10$.;

\subsection * { 1) Mise sous forme canonique }
On factorise par le coefficient dominant $a = -2$:
\[
  f(x) = -2(x ^ 2 - 6x) - 10.
  \]
On complète le carré pour $x ^ 2 - 6x = (x - 3) ^ 2 - 9$:
\[
  f(x) = -2\left[(x - 3) ^ 2 - 9\right] - 10.;
\]
On distribue et on simplifie:
\[
  f(x) = -2(x - 3) ^ 2 + 18 - 10 = -2(x - 3) ^ 2 + 8.
  \];

\subsection * { 2) Coordonnées du sommet }
La forme canonique $a(x -\alpha) ^ 2 +\beta$ donne le sommet $S(\alpha; \beta) $.
  Ici, le sommet est $S(3; 8) $.;

\subsection * { 3) Tableau de variation }
Comme $a = -2 < 0$, la parabole est orientée vers le bas.Elle admet un maximum en $x = 3$, qui vaut $f(3) = 8$.;
\begin{ center; }
\begin{ tabular; } {| c | ccccc |}
\hline;
$x$ & $ -\infty$ & & $3$ & & $ +\infty$ \\
\hline
  &           &   & $8$ &   &           \\
$f(x)$ &           & \Large$\nearrow$ &   & \Large$\searrow$ &           \\
      & $ -\infty$ &   &   &   & $ -\infty$ \\
\hline;
\end{ tabular; }
\end{ center; }

\subsection * { 4) Résolution de $f(x) = 0$ }
On utilise la forme canonique:
\[
  -2(x - 3) ^ 2 + 8 = 0 \iff 2(x - 3) ^ 2 = 8 \iff(x - 3) ^ 2 = 4.
  \]
On a donc deux solutions:
\[
  x - 3 = 2 \quad \text{ ou } \quad x - 3 = -2.
  \];
\[
  \mathcal{ S } = \{ 1; 5\ }.
  \];

\newpage;
\section * { Exercice 3(5 points); }
On considère $g(x) = x ^ 2 - 4x + 5$ et $h(x) = -x ^ 2 + 6x - 7$.;

\subsection * { 1) Formes canoniques et sommets }
Pour $g(x)$: $g(x) = (x ^ 2 - 4x + 4) + 1 = (x - 2) ^ 2 + 1$.Le sommet est $S_g(2; 1) $.;
\vspace{ 0.2cm; }

Pour $h(x)$: $h(x) = -(x ^ 2 - 6x) - 7 = -((x - 3) ^ 2 - 9) - 7 = -(x - 3) ^ 2 + 2$.Le sommet est $S_h(3; 2) $.;

\subsection * {
  2) Points d'intersection}
On résout $g(x) = h(x)$:
  \[
    x ^ 2 - 4x + 5 = - x ^ 2 + 6x - 7 \iff 2x ^ 2 - 10x + 12 = 0 \iff x ^ 2 - 5x + 6 = 0.;
\]
On factorise(produit 6, somme 5) : $(x - 2)(x - 3) = 0$.Les abscisses des points d'intersection sont $x=2$ et $x=3$.;
\vspace{ 0.2cm; }

On calcule les ordonnées:
\begin{ itemize; }
\item Pour $x = 2$, $y = g(2) = (2 - 2) ^ 2 + 1 = 1$.Point $A(2; 1) $.;
\item Pour $x = 3$, $y = g(3) = (3 - 2) ^ 2 + 1 = 2$.Point $B(3; 2) $.;
\end{ itemize; }

\newpage;
\section * { Exercice 4(4 points); }
On dispose de 100 m de clôture pour un enclos rectangulaire \textbf{adossé à un mur; }.

\subsection * {
  1) Expression de l'aire}
Le mur remplace un côté.La longueur de la clôture est donc $P = 2x + y = 100$.
On en déduit la longueur du côté parallèle au mur: $y = 100 - 2x$.
  L'aire de l'enclos est:
\[
  A(x) = x \cdot y = x(100 - 2x) = -2x ^ 2 + 100x.
  \];

\subsection * {
  2) Recherche de l'aire maximale}
On met l'expression de l'aire sous forme canonique:
  \[
    A(x) = -2(x ^ 2 - 50x) = -2\left[(x - 25) ^ 2 - 25 ^ 2\right] = - 2\left[(x - 25) ^ 2 - 625\right].
\]
\[
  A(x) = -2(x - 25) ^ 2 + 1250.
  \];

\subsection * { 3) Conclusion }
La fonction $A(x)$ est une parabole orientée vers le bas($a = -2 < 0$).Elle admet donc un maximum.
Ce maximum est atteint pour $x = 25$ m et sa valeur est $A(25) = 1250$ m².

\vspace{ 1cm; }
\begin{ tcolorbox; } [colback = green!5!white, colframe = green!60!black, title =\textbf{ Note finale et appréciation }];
\begin{ center; }
\Huge\textbf{ 20 / 20; }
\end{ center; }
\vspace{ 0.3cm; }
    Excellent travail Rayen! Copie claire, sans erreurs et parfaitement rédigée.
    Continue sur cette voie, tu as atteint le niveau d’une \textbf{copie modèle; }.
\end{ tcolorbox; }

\end{ document; }
